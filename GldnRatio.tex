%%%%%%%%%%%%%%%%%%%%%%%%%%%%%%%%%%%%%%%%%%%%%%%%%%%%%%%%%%%%%%%%
%% proto.tex: 原稿テンプレート
%% Based on `proto-one` 2019/04/04 [v2]
%% 2019/04/04 (sample)
%%%%%%%%%%%%%%%%%%%%%%%%%%%%%%%%%%%%%%%%%%%%%%%%%%%%%%%%%%%%%%%%
%%%% [build.rc]: [FILES_UPLATEX]
%%%% [build.rc]: [FILES_FULL3mm]
%%%%%%%%%%%%%%%%%%%%%%%%%%%%%%%%%%%%%%%%%%%%%%%%%%%%%%%%%%%%%%%%
%%%% [PDF.info]: [Title]: プロトタイプ原稿
%%%% [PDF.info]: [Author]: ●● ●●
%%%% [PDF.info]: #[Subject]: サブタイトル
%%%% [PDF.info]: #[Keywords]: キーワード1
%%%% [PDF.info]: #[Keywords]: キーワード2
%%%% [PDF.info]: [Creator]: 暗黒通信団 The Darkside Communication Group
%%%% [PDF.info]: #[Creator]: 神戸暗黒通信団 The Kobe Darkside Communication Group
%%%%%%%%%%%%%%%%%%%%%%%%%%%%%%%%%%%%%%%%%%%%%%%%%%%%%%%%%%%%%%%%
\documentclass[a5paper,dvipdfmx,uplatex]{jsarticle}
%%%%%%%%%%%%%%%%%%%%%%%%%%%%%%%%%%%%%%%%%%%%%%%%%%%%%%%%%%%%%%%%
\RequirePackage{etoolbox}
\IfFileExists{lgrenc.def}{\PassOptionsToPackage{LGR}{fontenc}}{}
\RequirePackage[T1]{fontenc}
\RequirePackage{lmodern}
%%%%%%%%%%%%%%%%%%%%%%%%%%%%%%%%%%%%%%%%%%%%%%%%%%%%%%%%%%%%%%%%
\IfFileExists{./dark-noto.sty}{
  \AfterEndDocument{\typeout{** ./dark-noto.sty あり: Source Han / Noto CJK は有効.}}
  \RequirePackage{dark-noto}
}{% ./dark-noto.sty がないとき
  \AfterEndDocument{\typeout{** ./dark-noto.sty なし: Source Han / Noto CJK は無効.}}
}
%%%%%%%%%%%%%%%%%%%%%%%%%%%%%%%%%%%%%%%%%%%%%%%%%%%%%%%%%%%%%%%%
\RequirePackage[small,okuduke,noload-okuduke,draftnotes]{dark-one}
%%%%%%%%%%%%%%%%%%%%%%%%%%%%%%%%%%%%%%%%%%%%%%%%%%%%%%%%%%%%%%%%
% 校正用コメントの設定
\IfFileExists{./release.do}{% ./release.do があるとき
  \AfterEndDocument{\typeout{** ./release.do あり: 校正用コメントは無効.}}
  \DisableDraftNotes
  \PassOptionsToPackage{nostamp}{draftwatermark}% draftwatermarkを無効化
}{% ./release.do がないとき
  \AfterEndDocument{\typeout{** ./release.do なし: 校正用コメントは有効.}}
  \EnableDraftNotes
  \AtEndDocument{\PrintDraftNotes}%
  \PassOptionsToPackage{stamp}{draftwatermark}% draftwatermarkを有効化
}
%%%%%%%%%%%%%%%%%%%%%%%%%%%%%%%%%%%%%%%%%%%%%%%%%%%%%%%%%%%%%%%%
% 裁ち落とし画像を入れる
\IfFileExists{./dark-edge.do}{% ./dark-edge.do があるとき有効
  \AfterEndDocument{\typeout{** ./dark-edge.do あり: 裁ち落とし画像は有効.}}%
  \RequirePackage{dark-edge}%
  %\def\TailEdgePrefix{images/edges/tailedge_}%
  %\TailEdgeOffset30mm\relax%
  %\setTailEdgeBegin{1}%
  %\setTailEdgeEnd{1000000}%
}{%
  \AfterEndDocument{\typeout{** ./dark-edge.do なし: 裁ち落とし画像は無効.}}%
}
%%%%%%%%%%%%%%%%%%%%%%%%%%%%%%%%%%%%%%%%%%%%%%%%%%%%%%%%%%%%%%%%
\IfFileExists{\jobname.ver}{\input{\jobname.ver}}{}
%%%%%%%%%%%%%%%%%%%%%%%%%%%%%%%%%%%%%%%%%%%%%%%%%%%%%%%%%%%%%%%%
\RequirePackage{geometry}
\newlength\nzw{\normalsize\global\nzw1zw\relax}% \normalsize の全角幅
\geometry{nohead,top=18mm,bottom=17mm}
\geometry{textwidth=40\nzw}%
\topskip10pt
%%%%%%%%%%%%%%%%%%%%%%%%%%%%%%%%%%%%%%%%%%%%%%%%%%%%%%%%%%%%%%%%
\RequirePackage{exscale}
\RequirePackage{url}
\RequirePackage{bm}
\RequirePackage{okumacro}
\RequirePackage{ascmac}
\RequirePackage{pxrubrica}
\RequirePackage{amsmath}
\RequirePackage{amssymb}
\RequirePackage{siunitx}
\RequirePackage{latexsym}
\RequirePackage{graphicx}
\RequirePackage{wrapfig}
\RequirePackage{fancybox}
%%%%%%%%%%%%%%%%%%%%%%%%%%%%%%%%%%%%%%%%%%%%%%%%%%%%%%%%%%%%%%%%
\RequirePackage{draftwatermark}% 背景の透かしがうるさいときはコメントアウト
%%%%%%%%%%%%%%%%%%%%%%%%%%%%%%%%%%%%%%%%%%%%%%%%%%%%%%%%%%%%%%%%
\pagestyle{plain}

\begin{document}

%%%% 「%」のあとは、コメントとして無視されます。
%%%% ファイルの編集:コメント

% \section{セクション}

% \subsection{サブセクション}

\IfFileExists{\jobname.tbl}{\input{\jobname.tbl}}{}

% \begin{itemize}
% \item 箇条書き(番号なし)
% \item 箇条書き(番号なし)
% \end{itemize}

% \begin{enumerate}
% \item 箇条書き(番号あり)
% \item 箇条書き(番号あり)
% \end{enumerate}

% \begin{description}
% \item[ラベル] 箇条書き(ラベルつき)
% \item[ラベル] 箇条書き(ラベルつき)
% \end{enumerate}

% \draftnote{校正用メモ}

% \appendix% これ以降、セクションの番号の付け方が変わります。
% \section{ふろく}


%%%% 参考文献:必要あれば
%% 下のようにしておけば、\cite{aaaa} などでその文献番号を表示できる。
% \begin{thebibliography}{99}
% \bibitem{aaaa} ...
% \bibitem{bbbb} ...
% \end{thebibliography}


%%%%%%%%%%%%%%%%%%%%%%%%%%%%%%%%%%%%%%%%%%%%%%%%%%%%%%%%%%%%%%%%
%% okuduke.tex: 奥付テンプレート
%% Based on `proto-one` 2019/04/04 [v2]
%% 2019/04/04 (sample)
%%%%%%%%%%%%%%%%%%%%%%%%%%%%%%%%%%%%%%%%%%%%%%%%%%%%%%%%%%%%%%%%
\def\okudukefootnotetext{}
\def\okudukeadjustmargin{}
\def\okudukeadjustmargin{\hspace*{1.5zw}}%% proto-one 標準 1.5zw
\iffalse%%% 最終ページに脚注があるとき、\iftrue に変更
  \def\okudukefootnotetext{\footnotetext[0]}
  \def\okudukeadjustmargin{\hspace*{-3zw}}
\fi
%%%%%%%%%%%%%%%%%%%%%%%%%%%%%%%%%%%%%%%%%%%%%%%%%%%%%%%%%%%%%%%%
%\clearpage
\null
\vfill
\okudukefootnotetext{%
%\okudukeskip
\noindent\okudukeadjustmargin%
%\resizebox{12cm}{!}%通常サイズの奥付にしたいときはこの行をコメントアウト
{%
\begin{okuduke}{7.5cm}%% proto-one 暗黒標準 7.5cm,神戸標準 8.75cm
%% ↑著作年表示が複数年にわたるときは 8cm 以上がよい
%%%%%%%%%%%%%%%%%%%%%%%%%%%%%%%%%%%%%%%%%%%%%%%%%%%%%%%%%%%%%%%%
%%%% 基本書誌情報
\oSetTitle{% 書名 \oTitle で出力する
% プロトタイプ原稿
\jruby[<g>]{{プロトタイプ原稿}}{ぷろとたいぷげんこう}%
}
\oSetDateLines{% 発行履歴: oDate環境(tabular環境)に渡される
%2018&年& 8&月&12&日& 初版 発行%\\% コミックマーケット94 東京ビッグサイト 東フ30b
%2018&年&12&月&31&日& 初版 発行%\\% コミックマーケット95 東京ビッグサイト 月曜東U37b 暗黒
%2018&年&12&月&31&日& 初版 発行%\\% コミックマーケット95 東京ビッグサイト 月曜東U37a 神戸
%2019&年& 4&月&14&日& 初版 発行%\\% 技術書典6
2019&年&??&月&??&日& 初版 発行%\\%
}
\oSetInfoLines{% 各種情報: oInfo環境(tabular環境)に渡される
著 者& ●● ●● {\tiny(●●●● ●●●●)} \\
発行者& 星野 香奈 {\tiny(ほしの かな)}\\
発行所& {\tiny 同人集合} 暗黒通信団 (\texttt{http://ankokudan.org/d/})\\
   & 〒277-8691 千葉県柏局私書箱54号 D係\\
% 発行者& 趙 逖董 {\tiny(ちょう てきとう)}\\% 神戸暗黒通信団
% 発行所& {\tiny 革命的同人主義者同盟} 神戸暗黒通信団 (\texttt{http://ankokudan.org/k/})\\% 神戸暗黒通信団
%    & 〒xxx-xxxx ○○県○○○○ \\% 神戸暗黒通信団 住所なし?
頒 価& \oPrice\ / ISBN\oISBN\ \oBunrui
}
\oSetCopyrightYear{2019}%% 発行年     \oCopyrightYear で参照
\oSetPrice{???円}%            % おねだん   \oPrice で参照
\oSetISBN{978-4-87310-???-?}   % ISBNコード \oISBN で参照
\oSetBunrui{C????}%            % 分類記号   \oBunrui で参照
\oSetLogoPhrase{% コメント:↓小ネタを入れて書き換える。
乱丁・落丁は在庫がある限りお取り替えいたします。%
}%
%%%%%%%%%%%%%%%%%%%%%%%%%%%%%%%%%%%%%%%%%%%%%%%%%%%%%%%%%%%%%%%%
%%%% 奥付用の基本設定
\baselineskip12pt\parskip0sp\scriptsize\bfseries\gtfamily%
\oTitle% \oSetTitle で設定した書名
\vskip2pt
\oruleheight1mm% 奥付用の太めの線の幅(高さ)
\orulefill% ■■■■■■■■奥付用の太めの線■■■■■■■■■■■■
\vskip0mm
\oDateLines% 発行履歴
\vskip0mm
\oInfoLines% 各種情報
\vskip0mm
\orulefill% ■■■■■■■■奥付用の太めの線■■■■■■■■■■■■
\vskip0mm
\oSetLogoWidth{1cm}% \oLogoWidth と \oLogoRemainderWidth を設定する
% \oLogoWidth: ロゴ部分の幅
% \oLogoRemainderWidth: コメント部分の幅 (\textwidth-\oLogoWidth)
\parbox{\oLogoWidth}{\makebox[\oLogoWidth][c]{\truecons}}%
{\parbox{\oLogoRemainderWidth}{\normalfont\scriptsize\oLogoPhrase}}% めいっぱい
%{\normalsize\parbox{19.5zw}{\normalfont\scriptsize\oLogoPhrase}}% すこし限定
\vskip1pt
\orulefill[0.1mm]%─────細い線─────────────────
\vskip0mm
\oLastLine{\scriptsize
%\copyleft Copyleft \oCopyrightYear\ 暗黒通信団% コピーレフトのときはこちら
\copyright Copyright \oCopyrightYear\ 暗黒通信団% コピーライト通常表示
%\copyleft Copyleft \oCopyrightYear\ 神戸暗黒通信団% コピーレフトのときはこちら
%\copyright Copyright \oCopyrightYear\ 神戸暗黒通信団% コピーライト通常表示
  \hfill Printed in Japan% WEB公開版のときはコメントアウトする
}
%%%%%%%%%%%%%%%%%%%%%%%%%%%%%%%%%%%%%%%%%%%%%%%%%%%%%%%%%%%%%%%%
\end{okuduke}%
}%
\vspace*{1mm}%% proto-one 標準 1mm
%\vfill%
%\null%
}
%%%%%%%%%%%%%%%%%%%%%%%%%%%%%%%%%%%%%%%%%%%%%%%%%%%%%%%%%%%%%%%%

\end{document}

%%%% $ FilePath: proto.tex $
%%%% $ FileHash: ec207a56a89cbcae91fba453805ba67e4ce6b090 $
%%%% $ FileAuthorDate: 2019-04-04T04:04:04+09:00 $
%%%% $ FileCommitDate: 2019-04-04T04:04:04+09:00 $
%%%% $ FileCommitHash: 3faab843b277137d8b621a01e1abf1ba1c5b2971 $
%%%% $ AuthorDate: 2019-04-04T04:04:04+09:00 $
%%%% $ CommitDate: 2019-04-04T04:04:04+09:00 $
%%%% $ CommitHash: 3faab843b277137d8b621a01e1abf1ba1c5b2971 $
%%%% $ ProjectName: proto-one $
%%%% $ RefName: v2_d20190404 $
%%%% $ Status: $
